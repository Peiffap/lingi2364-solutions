\documentclass[pdf]{article}
\usepackage[a4paper, margin=1in]{geometry}
\usepackage{helvet}
\usepackage{amsmath}
\usepackage{amssymb}
\usepackage{amsthm}
\usepackage{xspace}
\usepackage{url}
\renewcommand{\familydefault}{\sfdefault}
\usepackage[pdf]{pstricks}
%\usepackage{rotating}
\usepackage{pst-node}
\usepackage{forest}
\usepackage{mathrsfs}

\newcommand{\db}{\mathcal{D}}
\newcommand{\cover}[2]{\mathrm{cover}_{#1}({#2})}
\newcommand{\bc}{\{B, C\}}
\newcommand{\bd}{\{B, D\}}
\newcommand{\be}{\{B, E\}}
\newcommand{\seq}[1]{\langle #1 \rangle}
\newcommand{\coverseq}[1]{\mathrm{cover}(\seq{#1})}

\title{LINGI2364: Mining Patterns in Data\\\bf Exercise session 2: Depth-First Algorithms for Itemset Mining and Sequential Pattern Mining}
\date{25 February 2020}
\author{Gilles Peiffer}

\begin{document}

\maketitle
\setcounter{section}{1}
\section{Solutions}

\begin{enumerate}
	\item \begin{enumerate}
		\item For a minimum support threshold of 2, the projected databases are obtained as follows (by convention, transactions which do not appear are empty):
		\begin{align*}
			\db_{\vert \emptyset}(t) &= \left\{\begin{array}{l@{\quad}l}
			\db(t), & \textnormal{if } t \in \{0, \dots, 9\} \setminus \{7\}, \\
			\db(7) \setminus \{G\}, & \textnormal{otherwise}.
			\end{array} \right. \\\\
			\db_{\vert \{B\}}(2) &= \{D\}, \\
			\db_{\vert \{B\}}(3) &= \{C, D\}, \\
			\db_{\vert \{B\}}(4) &= \{C\}, \\
			\db_{\vert \{B\}}(5) &= \{D\}, \\
			\db_{\vert \{B\}}(6) &= \{D, E\}, \\
			\db_{\vert \{B\}}(7) &= \{C, D, E\}, \\
			\db_{\vert \{B\}}(9) &= \{D\}. \\\\
			\db_{\vert \{B, C\}}(3) &= \{D\}, \\
			\db_{\vert \{B, C\}}(7) &= \{D\}.
		\end{align*}
		\item \begin{align*}
			\cover{\db_{\vert \emptyset}}{\{A\}} &= \{0, 1, 5, 9\}, \\
			\cover{\db_{\vert \emptyset}}{\{B\}} &= \{2, 3, 4, 5, 6, 7, 9\}, \\
			\cover{\db_{\vert \emptyset}}{\{C\}} &= \{1, 3, 4, 7, 8\}, \\
			\cover{\db_{\vert \emptyset}}{\{D\}} &= \{0, 1, 2, 3, 5, 6, 7, 8, 9\}, \\
			\cover{\db_{\vert \emptyset}}{\{E\}} &= \{1, 6, 7\}, \\
			\cover{\db_{\vert \emptyset}}{\{F\}} &= \{0, 8\}. \\\\
			\cover{\db_{\vert \{B\}}}{\{C\}} &= \{3, 4, 7\}, \\
			\cover{\db_{\vert \{B\}}}{\{D\}} &= \{2, 3, 5, 6, 7, 9\}, \\
			\cover{\db_{\vert \{B\}}}{\{E\}} &= \{6, 7\}. \\\\
			\cover{\db_{\vert \{B, C\}}}{\{D\}} &= \{3, 7\}.
		\end{align*}
		\item At node \(\{A\}\), the projected database is
		\begin{align*}
			\cover{\db_{\vert \{A\}}}{\{B\}} &= \{5, 9\}, \\
			\cover{\db_{\vert \{A\}}}{\{D\}} &= \{0, 1, 5, 9\}.
		\end{align*}
		The DFS algorithm next explores node \(\{A, B\}\), which has the following projected database:
		\begin{align*}
			\cover{\db_{\vert \{A, B\}}}{\{D\}} &= \{5, 9\}.
		\end{align*}
		However, \(\cover{\db_{\vert \{A, B, D\}}}{I} = \emptyset\), for any item \(I\).
		
		The search thus backtracks, and explores node \(\{A, D\}\).
		\(\cover{\db_{\vert \{A, D\}}}{I} = \emptyset\), for any item \(I\), hence the search backtracks, and then once again, back to the root node.
		This concludes the DFS algorithm.
		\item The inverted tid-list for \(\{A\}\) is \(\{2, 3, 4, 6, 7, 8\}\).
		The inverted tid-list for \(\{B\}\) is \(\{0, 1, 8\}\).
		The inverted tid-list for \(\{A, B\}\) can then be computed by taking the union of the previously identified inverted tid-lists for \(\{A\}\) and \(\{B\}\): \(\{0, 1, 2, 3, 4, 6, 7, 8\}\).
		
		Inverted tid-lists have the advantage that computing a set union is easier than computing an intersection.
		However, computing the cover is harder, and has higher memory requirements, since very quickly, inverted tid-lists are close to \(\mathcal{T}\).
	\end{enumerate}
	\item \begin{enumerate}
		\item The FP-tree for \(\emptyset\) is the following:
		
		\begin{center}
			\begin{forest}
				for tree={
					parent anchor=south,
					child anchor=north,
					fit=band,% spaces the tree out a little to avoid collisions
				}
				[x, phantom
					[Header Table
						[\(F : 2\), no edge, tier=f, name=fhead
							[\(E : 3\), tier=e, no edge, name=ehead
								[\(D : 9\), tier=d, no edge, name=dhead
									[\(C : 5\), tier=c, no edge, name=chead
										[\(B : 7\), tier=b, no edge, name=bhead
											[\(A : 4\), tier=a, no edge, name=ahead]
										]
									]
								]
							]
						]
					]
					[\(\emptyset\), no edge, name=root
						[\(F : 2\), tier=f, name=f1
							[\(D : 2\), tier=d, name=d1
								[\(C : 1\), tier=c, name=c1]
								[\(A : 1\), tier=a, name=a1]
							]
						]
						[\(E : 3\), tier=e, name=e1
							[\(D : 3\), tier=d, name=d2
								[\(C : 2\), tier=c, name=c2
									[\(A : 1\), tier=a, name=a2]
									[\(B: 1\), tier=b, name=b1]
								]
								[\(B : 1\), tier=b, name=b2]
							]
						]
						[\(D : 4\), tier=d, name=d3
							[\(C : 1\), tier=c, name=c3
								[\(B : 1\), tier=b, name=b3]
							]
							[\(B : 3\), tier=b, name=b4
								[\(A : 2\), tier=a, name=a3]
							]
						]
						[\(C : 1\), tier=c, name=c4
							[\(B : 1\), tier=b, name=b5]
						]
					]
				]
				\draw [red, ->] (fhead.east) -- (f1.west);
				\draw [green, ->] (ehead.east) -- (e1.west);
				\draw [blue, ->] (dhead.east) -- (d1.west);
				\draw [blue, ->] (d1.east) -- (d2.west);
				\draw [blue, ->] (d2.east) -- (d3.west);
				\draw [orange, ->] (chead.east) -- (c1.west);
				\draw [orange, ->] (c1.east) -- (c2.west);
				\draw [orange, ->] (c2.east) -- (c3.west);
				\draw [orange, ->] (c3.east) -- (c4.west);
				\draw [purple, ->] (bhead.east) -- (b1.west);
				\draw [purple, ->] (b1.east) -- (b2.west);
				\draw [purple, ->] (b2.east) -- (b3.west);
				\draw [purple, ->] (b3.east) -- (b4.west);
				\draw [purple, ->] (b4.east) -- (b5.west);
				\draw [yellow, ->] (ahead.east) -- (a1.west);
				\draw [yellow, ->] (a1.east) -- (a2.west);
				\draw [yellow, ->] (a2.east) -- (a3.west);
			\end{forest}
		\end{center}
		
		The FP-tree for \(\{B\}\) is the following:
		
		\begin{center}
			\begin{forest}
				for tree={
					parent anchor=south,
					child anchor=north,
					fit=band,% spaces the tree out a little to avoid collisions
				}
				[x, phantom
					[Header Table
						[\(E : 2\), tier=e, no edge, name=ehead
							[\(D : 6\), tier=d, no edge, name=dhead
								[\(C : 3\), tier=c, no edge, name=chead]
							]
						]
					]
					[\(\{B\}\), no edge, name=root
						[\(E : 2\), tier=e, name=e1
							[\(D : 2\), tier=d, name=d1
								[\(C : 1\), tier=c, name=c1]
							]
						]
						[\(D : 4\), tier=d, name=d2
							[\(C : 1\), tier=c, name=c2]
						]
						[\(C : 1\), tier=c, name=c3]
					]
				]
				\draw [green, ->] (ehead.east) -- (e1.west);
				\draw [blue, ->] (dhead.east) -- (d1.west);
				\draw [blue, ->] (d1.east) -- (d2.west);
				\draw [orange, ->] (chead.east) -- (c1.west);
				\draw [orange, ->] (c1.east) -- (c2.west);
				\draw [orange, ->] (c2.east) -- (c3.west);
			\end{forest}
		\end{center}
		
		The FP-tree for \(\{B, C\}\) is the following:
		
		\begin{center}
			\begin{forest}
				for tree={
					parent anchor=south,
					child anchor=north,
					fit=band,% spaces the tree out a little to avoid collisions
				}
				[x, phantom
					[Header Table
						[\(D : 2\), tier=d, no edge, name=dhead]
					]
					[\(\bc\), no edge, name=root % weird workaround, idfk
						[\(D : 2\), tier=d, name=d1]
					]
				]
				\draw [blue, ->] (dhead.east) -- (d1.west);
			\end{forest}
		\end{center}
		\item By sorting on the frequencies instead of alphabetically (\(D \to C \to E\)), one finds the following FP-tree for \(\{B\}\):
		
		\begin{center}
			\begin{forest}
				for tree={
					parent anchor=south,
					child anchor=north,
					fit=band,% spaces the tree out a little to avoid collisions
				}
				[x, phantom
					[Header Table
						[\(D : 6\), tier=d, no edge, name=dhead
							[\(C : 3\), tier=c, no edge, name=chead
								[\(E : 2\), tier=e, no edge, name=ehead]
							]
						]
					]
					[\(\{B\}\), no edge, name=root
						[\(D : 6\), tier=d, name=d1
							[\(C : 2\), tier=c, name=c1
								[\(E : 1\), tier=e, name=e1]
							]
							[\(E: 1\), tier=e, name=e2]
						]
						[\(C : 1\), tier=c, name=c2]
					]
				]
				\draw [blue, ->] (dhead.east) -- (d1.west);
				\draw [orange, ->] (chead.east) -- (c1.west);
				\draw [orange, ->] (c1.east) -- (c2.west);
				\draw [green, ->] (ehead.east) -- (e1.west);
				\draw [green, ->] (e1.east) -- (e2.west);
			\end{forest}
		\end{center}
		\item The first node to be explored is \(\{B, D\}\).
		However, its FP-tree is empty.
		The search backtracks, and visits \(\{B, C\}\), which has FP-tree
		\begin{center}
			\begin{forest}
				for tree={
					parent anchor=south,
					child anchor=north,
					fit=band,% spaces the tree out a little to avoid collisions
				}
				[x, phantom
					[Header Table
						[\(D : 2\), tier=d, no edge, name=dhead]
					]
					[\(\bc\), no edge, name=root % weird workaround, idfk
						[\(D : 2\), tier=d, name=d1]
					]
				]
				\draw [blue, ->] (dhead.east) -- (d1.west);
			\end{forest}
		\end{center}
		The search then continues down its path, towards \(\{B, C, D\}\), which has an empty FP-tree.
		The search backtracks twice, then visits \(\{B, E\}\), with FP-tree
		\begin{center}
			\begin{forest}
				for tree={
					parent anchor=south,
					child anchor=north,
					fit=band,% spaces the tree out a little to avoid collisions
				}
				[x, phantom
					[Header Table
						[\(D : 2\), tier=d, no edge, name=dhead]
					]
					[\(\be\), no edge, name=root % weird workaround, idfk
						[\(D : 2\), tier=d, name=d1]
					]
				]
				\draw [blue, ->] (dhead.east) -- (d1.west);
			\end{forest}
		\end{center}
		The next child of this node, \(\{B, D, E\}\), has an empty FP-tree.
		The search backtracks all the way up to the root node, and the algorithm terminates.
	\end{enumerate}
	\item \begin{enumerate}
		\item \begin{align*}
			\coverseq{\{A, B\}, \{C\}} &= \{0, 1\}, \\
			\coverseq{\{A, B\}, \{D\}} &= \{0, 1\}, \\
			\coverseq{\{A\}, \{B\}} &= \{2, 3\}, \\
			\coverseq{\{A\}, \{D\}} &= \{0, 1, 2\}. \\
		\end{align*}
		\item The frequent sequences are:
		\begin{itemize}
			\item \(\seq{\{A\}}, \seq{\{B\}}, \seq{\{C\}}, \seq{\{D\}}\),
			\item \(\seq{\{A, B\}}, \seq{\{A, C\}}\),
			\item \(\seq{\{A\}, \{A\}}\), \(\seq{\{A\}, \{B\}}\), \(\seq{\{A\}, \{C\}}\), \(\seq{\{A\}, \{D\}}\), \(\seq{\{B\}, \{C\}}\), \(\seq{\{B\}, \{D\}}\), \(\seq{\{C\}, \{A\}}\), \(\seq{\{C\}, \{B\}}\), \(\seq{\{C\}, \{D\}}\),
			\item \(\seq{\{A, B\}, \{C\}}, \seq{\{A, B\},\{D\}}\),
			\item \(\seq{\{A\}, \{A, C\}}\).
		\end{itemize}
		\item Let \(F = \langle I_1,\dots,I_n \rangle \in (2^\mathcal{I})^*\) be a frequent sequence.
		By anti-monotonicity, the sequence \(S = \langle J_1,\dots,J_m \rangle \in (2^\mathcal{I})^*\) is also frequent, if there exist integers \(1 \leq i_1 < \dots < i_m \leq n\) such that for all \(j\in\{1,\ldots,m\} : J_j \subseteq I_{i_j}\).
		
		Conversely, let \(I = \langle I_1,\dots,I_n \rangle \in (2^\mathcal{I})^*\) be an infrequent sequence.
		By anti-monotonicity, the sequence \(S = \langle J_1,\dots,J_m \rangle \in (2^\mathcal{I})^*\) is then also infrequent, if there exist integers \(1 \leq j_1 < \dots < j_n \leq m\) such that for all \(i\in\{1,\ldots,n\} : I_i \subseteq J_{j_i}\).
		
		This formalizes the intuition of anti-monotonicity: ``If a sequence is frequent, all sequences which are contained in it are also frequent; if a sequence is infrequent, all sequences in which it is contained are also infrequent.''
		\item No, there is no anti-monotonicity property.
		\begin{proof}
			By contradiction.
			Take the single transaction database \(\seq{\{A\}, \{B\}, \{B\}}\).
			For this database, \(\mathrm{support}(\seq{\{A\}}) = 1 < \mathrm{support}(\seq{\{A\}\{B\}}) = 2\), despite the fact that \(\seq{\{A\}} \subseteq \seq{\{A\}\{B\}}\).
		\end{proof}
	\end{enumerate}
	\item \begin{enumerate}
		\item Let the relation be written as \(\preceq\). \(\preceq\) is a partial order if and only if the three following properties are satisfied:
		\begin{itemize}
			\item Reflexivity: \(a \preceq a\).
			\item Antisymmetry: if \(a \preceq b\) and \(b \preceq a\), then \(a = b\).
			\item Transitivity: if \(a \preceq b\) and \(b \preceq c\), then \(a \preceq c\).
		\end{itemize}
		\item We assume all elements are different.
		This relation is then not a partial order.
		\begin{proof}
			By contradiction.
			Since \(A \succeq C\) and \(C \succeq D\), by transitivity one would obtain \(A \succeq D\), however this is not the case.
		\end{proof}
		\item We do not prove it formally, but \textit{match} is a partial order, since it (intuitively) verifies all three required properties.
		\item This constraint is a gap constraint.
		It expresses the fact that the number of itemsets between matched itemsets in a patter must not exceed some limit \(g - 1\).
		\item No, it is no longer a partial order.
		\begin{proof}
			By contradiction.
			Take \(X = \seq{\{A\}, \{D\}}, Y = \seq{\{A\}, \{C\}, \{D\}}, Z = \seq{\{A\}, \{B\}, \{C\}, \{D\}}\), with \(g=2\).
			It is easy to verify that \(\mathit{match}(X, Y)\) and \(\mathit{match}(Y, Z)\), however, \(\lnot \mathit{match}(X, Z)\) since the gap between \(\{A\}\) and \(\{D\}\) in \(Z\) is \(3 > g = 2\).
		\end{proof}
		\item With the length constraint, \textit{match} is still a partial order.
		
		When combining the two constraints, \textit{match} is a partial order if \(g \ge \ell\), as this makes the gap constraint redundant.
	\end{enumerate}
	\item \begin{enumerate}
		\item \begin{itemize}
			\item At level 1, we generate \(\seq{\{A\}}\), \(\seq{\{B\}}\), \(\seq{\{C\}}\), \(\seq{\{D\}}\).
			All are frequent.
			\item At level 2, we generate \(\seq{\{A, B\}}\), \(\seq{\{A, C\}}\), \(\seq{\{A, D\}}\), \(\seq{\{A\}, \{A\}}\), \(\seq{\{A\}, \{B\}}\), \(\seq{\{A\}, \{C\}}\), \(\seq{\{A\}, \{D\}}\), \(\seq{\{B\}, \{A\}}\), \(\seq{\{B\}, \{B\}}\), \(\seq{\{B\}, \{C\}}\), \(\seq{\{B\}, \{D\}}\), \(\seq{\{C\}, \{A\}}\),  \(\seq{\{C\}, \{B\}}\), \(\seq{\{C\}, \{C\}}\), \(\seq{\{C\}, \{D\}}\), \(\seq{\{D\}, \{A\}}\), \(\seq{\{D\}, \{B\}}\), \(\seq{\{D\}, \{C\}}\), \(\seq{\{D\}, \{D\}}\).
			
			The frequent sequences are \(\seq{\{A, B\}}\), \(\seq{\{A, C\}}\), \(\seq{\{A\}, \{A\}}\), \(\seq{\{A\}, \{B\}}\), \(\seq{\{A\}, \{C\}}\), \(\seq{\{A\}, \{D\}}\), \(\seq{\{B\}, \{C\}}\), \(\seq{\{B\}, \{D\}}\), \(\seq{\{C\}, \{A\}}\), \(\seq{\{C\}, \{B\}}\), \(\seq{\{C\}, \{D\}}\).
			\item At level 3, we generate \(\seq{\{A, B, C\}}\), \(\seq{\{A, B\}, \{A\}}\), \(\seq{\{A, B\}, \{B\}}\), \(\seq{\{A, B\}, \{C\}}\), \(\seq{\{A, B\}, \{D\}}\), \(\seq{\{A, C\}, \{A\}}\), \dots
			
			The frequent sequences are \(\seq{\{A, B\}, \{C\}}, \seq{\{A, B\},\{D\}}\), \(\seq{\{A\}, \{A, C\}}\).
			\item At level 4, we generate \(\seq{\{A, B\}, \{C, D\}}\), \(\seq{\{A, B\}, \{C\}, \{D\}}\), \(\seq{\{A, B\}, \{D\}, \{C\}}\).
			
			There are no more frequent sequences, and the GSP algorithm terminates.
		\end{itemize}
		Once this is done, we can eliminate all sequences which do not contain the item \(A\).
		\item We denote \(X\) being added to the last itemset of a sequence as branching on \(X\), and a new itemset being added starting with \(X\) by branching on \(\vert X\).
		The trie of frequent sequences looks like this:
		\begin{center}
			\begin{forest}
				for tree={
					parent anchor=south,
					child anchor=north,
					fit=band,% spaces the tree out a little to avoid collisions
				}
				[root
						[\(\vert A\)
							[\(\vert A\)
								[\(C\)]
							]
							[\(\vert B\)]
							[\(B\)
								[\(\vert C\)]
								[\(\vert D\)]
							]
							[\(\vert C\)]
							[\(C\)]
							[\(\vert D\)]
						]
						[\(\vert B\)
							[\(\vert C\)]
							[\(\vert D\)]
						]
						[\(\vert C\)
							[\(\vert A\)]
							[\(\vert B\)]
							[\(\vert D\)]
						]
						[\(\vert D\)]
				]
			\end{forest}
		\end{center}
	\end{enumerate}
\end{enumerate}

\end{document}
